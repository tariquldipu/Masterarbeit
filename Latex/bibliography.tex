\fancyhead[CE, CO]{\leftmark}
\setlength\parindent{0pt}
[1] Kermack, W. O., \& McKendrick, A. G. (1927). A contribution to the mathematical theory of epidemics. Proceedings of the royal society of london. Series A, Containing papers of a mathematical and physical character, 115(772), 700-721. \\
URL- \url{https://doi.org/10.1098/rspa.1927.0118} \\

[2] Definition of Zombie- Oxford Dictionary \\
URL- \url{https://www.oxfordlearnersdictionaries.com/definition/english/zombie?q=Zombie} \\
{[online, accessed 31-August-2021]}. \\

[3] Martcheva, M. (2015). An introduction to mathematical epidemiology (Vol. 61). New York: Springer. \\
URL- \url{https://doi.org/10.1007/978-1-4899-7612-3} \\

[4] Allen, L. J., Brauer, F., Van den Driessche, P., \& Wu, J. (2008). Mathematical epidemiology (Vol. 1945). Berlin: Springer. \\
URL- \url{https://www.springer.com/gp/book/9783540789109} \\

[5] Murray, J. D. (2002). Mathematical Biology I. An Introduction. JD Murray. New York: Springer. \\
URL- \url{https://doi.org/10.1007/b98868} \\

[6] Murray, J. D. (2001). Mathematical Biology II: spatial models and biomedical applications (Vol. 3). New York: Springer. \\
URL- \url{https://www.springer.com/gp/book/9780387952284} \\

[7] Hethcote, H. W., \& Van den Driessche, P. (1991). Some epidemiological models with nonlinear incidence. Journal of Mathematical Biology, 29(3), 271-287. \\
URL- \url{https://doi.org/10.1007/BF00160539} \\

[8] Ruan, S., \& Wang, W. (2003). Dynamical behavior of an epidemic model with a nonlinear incidence rate. Journal of Differential Equations, 188(1), 135-163. \\
URL- \url{https://www.sciencedirect.com/science/article/pii/S002203960200089X} \\

[9] Hethcote, H. W., Stech, H. W., \& Van Den Driessche, P. (1981). Stability analysis for models of diseases without immunity. Journal of Mathematical Biology, 13(2), 185-198. \\
URL- \url{https://doi.org/10.1007/BF00275213} \\

[10] Li, M. Y., \& Muldowney, J. S. (1995). Global stability for the SEIR model in epidemiology. Mathematical biosciences, 125(2), 155-164. \\
URL- \url{https://doi.org/10.1016/0025-5564(95)92756-5} \\

[11] Lofgren, E. T., Collins, K. M., Smith, T. C., \& Cartwright, R. A. (2016). Equations of the end: teaching mathematical modeling using the zombie apocalypse. Journal of microbiology \& biology education, 17(1), 137-142. \\
URL- \url{https://doi.org/10.1128/jmbe.v17i1.1066} \\

[12] Witkowski, C., \& Blais, B. (2013). Bayesian analysis of epidemics-zombies, influenza, and other diseases. arXiv preprint arXiv:1311.6376. \\
URL- \url{https://arxiv.org/abs/1311.6376} \\

[13] Cartmel, A. (2014). Mathematical modelling of zombies. University of Ottawa Press. \\
URL- \url{https://press.uottawa.ca/mathematical-modelling-of-zombies.html} \\

[14] Alemi, A. A., Bierbaum, M., Myers, C. R., \& Sethna, J. P. (2015). You can run, you can hide: the epidemiology and statistical mechanics of zombies. Physical Review E, 92(5), 052801. \\
URL- \url{https://doi.org/10.1103/PhysRevE.92.052801} \\

[15] Idu, E. I., \& Oladele, R. O. (2010). An Epidemic of Zombie Infection: A mathematical Model. University Of Ilorin, Nigeria. \\
URL- \url{https://www.sebbu.fr/zombies/Zombie_paper.pdf} \\

[16] Tomotani, J. V. (2015). Modeling and analysis of a Zombie Apocalypse: zombie infestation and combat strategies. Universidade de Sao Paolo, Brazil. \\
URL- \url{https://jgeekstudies.org/2015/05/18/zombie-model/} \\

[17] Munz, P., Hudea, I., Imad, J., \& Smith, R. J. (2009). When zombies attack!: mathematical modelling of an outbreak of zombie infection. Infectious disease modelling research progress, 4, 133-150. \\
URL- \url{https://webspace.science.uu.nl/~frank011/Classes/modsim/Handouts/Zombies.pdf} \\

[18] Allen, R. F., Jens, C., \& Wendt, T. J. (2014). Perturbations in Epidemiological Models. Letters in Biomathematics, 1(2), 173-180. \\
URL- \url{https://doi.org/10.1080/23737867.2014.11414478} \\

[19] Incubation Period- Wikipedia \\
URL- \url{https://en.wikipedia.org/wiki/Incubation_period} \\
{[online, accessed 04-September-2021]}. \\

[20] Definition of Homogeneity- Wikipedia  \\
URL- \url{https://en.wikipedia.org/wiki/Homogeneity_and_heterogeneity} \\
{[online, accessed 04-September-2021]}. \\

[21] SIR Model of Epidemics- Basic Model and Examples, Revised September 22, 2005, University of Rochester \\
URL- \url{http://www2.me.rochester.edu/courses/ME406/webexamp5/sir1.pdf} \\
{[online, accessed 04-September-2021]}. \\

[22] British Medical Journal, March 4 1978, p. 587 \\
URL- \url{https://www.ncbi.nlm.nih.gov/pmc/articles/PMC1603269/pdf/brmedj00115-0064.pdf} \\
{[online, accessed 04-September-2021]}. \\

[23] Python Scipy Library- solve\_ivp function \\
URL- \url{https://docs.scipy.org/doc/scipy/reference/generated/scipy.integrate.solve_ivp.html} \\
{[online, accessed 04-September-2021]}. \\

[24] Python Matplotlib \\
URL- \url{https://matplotlib.org} \\
{[online, accessed 04-September-2021]}. \\

[25] Python Numpy Library \\
URL- \url{https://numpy.org} \\
{[online, accessed 04-September-2021]}. \\

[26] Brauer, F., Castillo-Chavez, C., \& Castillo-Chavez, C. (2012). Mathematical models in population biology and epidemiology (Vol. 2, p. 508). New York: Springer. \\
URL- \url{https://doi.org/10.1007/978-1-4614-1686-9} \\

[27] Zerizer, T. (2006). Perturbation method for linear difference equations with small parameters. Advances in Difference Equations, 2006, 1-12. \\
URL- \url{https://link.springer.com/content/pdf/10.1155/ADE/2006/19214.pdf} \\

[28] Perturbation Analysis, San Jose State University \\
URL- \url{https://www.sjsu.edu/faculty/watkins/perturbation0.htm} \\
{[online, accessed 05-September-2021]}. \\

[29] O'Malley Jr, R. E. (1968). Topics in singular perturbations. Advances in Mathematics, 2(4), 365-470. \\
URL- \url{https://doi.org/10.1016/0001-8708(68)90023-6} \\

[30] O'MALLEY, R. E. (1967). Two-parameter singular perturbation problems for second-order equations. Journal of Mathematics and Mechanics, 16(10), 1143-1164. \\
URL- \url{https://www.jstor.org/stable/45277141} \\

[31] O'Malley Jr, R. E. (1991). Singular perturbation methods for ordinary differential equations. Springer-Verlag. \\
URL- \url{https://dl.acm.org/doi/abs/10.5555/116435} \\

[32] Shivamoggi, B. (2002). Perturbation methods for differential equations. Springer Science \& Business Media. \\
URL- \url{https://www.springer.com/gp/book/9780817641894} \\

[33] Perturbation theory (dynamical systems), Scholarpedia \\
URL- \url{http://www.scholarpedia.org/article/Perturbation_theory_(dynamical_systems)} \\
{[online, accessed 05-September-2021]}. \\

[34] Phase Portrait- Wikipedia \\
URL- \url{https://en.wikipedia.org/wiki/Phase_portrait} \\
{[online, accessed 06-September-2021]}. \\

[35] Differential Equations- Phase Plane, Paul's Online Notes \\
URL- \url{https://tutorial.math.lamar.edu/classes/de/phaseplane.aspx} \\
{[online, accessed 06-September-2021]}. \\

[36] Phase Portrait- Wolfram MathWorld \\
URL- \url{https://mathworld.wolfram.com/PhasePortrait.html} \\
{[online, accessed 06-September-2021]}. \\

[37] The Phase Plane- Zachary S Tseng, Pennsylvania State University \\
URL- \url{http://www.personal.psu.edu/sxt104/class/Math251/Notes-PhasePlane.pdf} \\
{[online, accessed 06-September-2021]}. \\

[38] Huang, X. C., \& Villasana, M. (2005). An extension of the Kermack-McKendrick model for AIDS epidemic. Journal of the Franklin Institute, 342(4), 341-351. \\
URL- \url{https://doi.org/10.1016/j.jfranklin.2004.11.008} \\

[39] Hu, Z., Ma, W., \& Ruan, S. (2012). Analysis of SIR epidemic models with nonlinear incidence rate and treatment. Mathematical biosciences, 238(1), 12-20. \\
URL- \url{https://doi.org/10.1016/j.mbs.2012.03.010} \\
	
[40] Meng, X., \& Chen, L. (2008). The dynamics of a new SIR epidemic model concerning pulse vaccination strategy. Applied Mathematics and Computation, 197(2), 582-597. \\
URL- \url{https://doi.org/10.1016/j.amc.2007.07.083} \\

[41] Bj{\o}rnstad, O. N., Finkenst\"adt, B. F., \& Grenfell, B. T. (2002). Dynamics of measles epidemics: estimating scaling of transmission rates using a time series SIR model. Ecological monographs, 72(2), 169-184. \\
URL- \url{https://doi.org/10.1890/0012-9615(2002)072[0169:DOMEES]2.0.CO;2} \\

[42] Smith, D., \& Moore, L. (2004). The SIR model for spread of disease-the differential equation model. Convergence. \\
URL- \url{http://www.geofhagopian.net/CS007A/CS7A-S20/The%20SIR%20Model%20for%20Spread%20of%20Disease%20-%20MAA.pdf} \\

[43] Weiss, H. H. (2013). The SIR model and the foundations of public health. Materials matematics, 0001-17. \\
URL- \url{https://ddd.uab.cat/record/108432} \\

[44] Cooper, I., Mondal, A., \& Antonopoulos, C. G. (2020). A SIR model assumption for the spread of COVID-19 in different communities. Chaos, Solitons \& Fractals, 139, 110057. \\
URL- \url{https://doi.org/10.1016/j.chaos.2020.110057} \\

[45] Chen, Y., Lu, P., \& Chang, C. (2020). A Time-dependent SIR model for COVID-19. Europe PMC. \\
URL- \url{https://europepmc.org/article/ppr/ppr345955} \\

[46] Python Matplotlib- Streamplot \\
URL- \url{https://matplotlib.org/stable/gallery/images_contours_and_fields/plot_streamplot.html} \\
{[online, accessed 06-September-2021]}. \\

[47] Tkinter- Python Library For GUI \\
URL- \url{https://docs.python.org/3/library/tkinter.html} \\
{[online, accessed 07-September-2021]}. \\

[48] Basic Reproduction Number- Wikipedia \\
URL- \url{https://en.wikipedia.org/wiki/Basic_reproduction_number} \\
{[online, accessed 11-September-2021]}. \\

[49] Basic Reproduction Number- Health Knowledge \\
URL- \url{https://www.healthknowledge.org.uk/public-health-textbook/research-methods/1a-epidemiology/epidemic-theory} \\
{[online, accessed 11-September-2021]}. \\

[50] Basic Reproduction Number- James Holland Jones (2007), Department of Anthropological Sciences, Standford University \\
URL- \url{https://web.stanford.edu/~jhj1/teachingdocs/Jones-on-R0.pdf} \\
{[online, accessed 11-September-2021]}. \\

[51] Dietz, K. (1993). The estimation of the basic reproduction number for infectious diseases. Statistical methods in medical research, 2(1), 23-41. \\
URL- \url{https://doi.org/10.1177%2F096228029300200103} \\

[52] Van den Driessche, P., \& Watmough, J. (2008). Further notes on the basic reproduction number. In Mathematical epidemiology (pp. 159-178). Springer, Berlin, Heidelberg. \\
URL- \url{https://doi.org/10.1007/978-3-540-78911-6_6} \\

[53] Spectral radius- Wikipedia \\
URL- \url{https://en.wikipedia.org/wiki/Spectral_radius} \\
{[online, accessed 11-September-2021]}. \\

[54] Jacobian Matrix- Wikipedia \\
URL- \url{https://en.wikipedia.org/wiki/Jacobian_matrix_and_determinant} \\
{[online, accessed 11-September-2021]}. \\

[55] Jacobian Matrix- Wolfram MathWorld \\
URL- \url{https://mathworld.wolfram.com/Jacobian.html} \\
{[online, accessed 11-September-2021]}. \\

[56] Upper Triangular Matrix- Wikipedia \\
URL- \url{https://en.wikipedia.org/wiki/Triangular_matrix} \\
{[online, accessed 11-September-2021]}. \\

[57] Diekmann, O., Heesterbeek, J. A. P., \& Roberts, M. G. (2010). The construction of next-generation matrices for compartmental epidemic models. Journal of the Royal Society Interface, 7(47), 873-885. \\
URL- \url{https://doi.org/10.1098/rsif.2009.0386} \\

[58] Heffernan, J. M., Smith, R. J., \& Wahl, L. M. (2005). Perspectives on the basic reproductive ratio. Journal of the Royal Society Interface, 2(4), 281-293. \\
URL- \url{https://doi.org/10.1098/rsif.2005.0042} \\

[59] Diekmann, O., \& Heesterbeek, J. A. P. (2000). Mathematical epidemiology of infectious diseases: model building, analysis and interpretation (Vol. 5). John Wiley \& Sons. \\
URL- \url{https://bit.ly/3u4GQN2} \\

[60] Heesterbeek, J. A. P. (2002). A brief history of R 0 and a recipe for its calculation. Acta biotheoretica, 50(3), 189-204. \\
URL- \url{https://doi.org/10.1023/A:1016599411804} \\

[61] Dormand, J. R., \& Prince, P. J. (1980). A family of embedded Runge-Kutta formulae. Journal of computational and applied mathematics, 6(1), 19-26. \\
URL- \url{https://doi.org/10.1016/0771-050X(80)90013-3} \\

[62] Butcher, J. C. (1996). A history of Runge-Kutta methods. Applied numerical mathematics, 20(3), 247-260. \\
URL- \url{https://doi.org/10.1016/0168-9274(95)00108-5} \\

[63] Shampine, L. F. (1986). Some practical runge-kutta formulas. Mathematics of computation, 46(173), 135-150. \\
URL- \url{https://doi.org/10.1090/S0025-5718-1986-0815836-3} \\

[64] Python Programming Language- Wikipedia\\
URL- \url{https://en.wikipedia.org/wiki/Python_(programming_language)}\\
{[online, accessed 28-September-2021]}.\\

[65] Python Scipy Library- odeint function\\
URL- \url{https://docs.scipy.org/doc/scipy/reference/generated/scipy.integrate.odeint.html}\\
{[online, accessed 28-September-2021]}.\\

[66] Overflow Error- Techopedia\\
URL- \url{https://www.techopedia.com/definition/663/overflow-error}\\
{[online, accessed 28-September-2021]}.\\