All the codes for this thesis work were written using Python programming language [64], version 3.7. This language was chosen as it provides sophisticated tools and libraries for a fast and convenient mathematical analysis and visualization. All the graphs were plotted using the Matplotlib [24] library. And all the differential equations were solved using the solve\_ivp function [23]. This function uses Runge-Kutta method of order five (RK45) [61], [62], [63] by default, as the integration method. In this RK45 method, the error is controlled assuming the accuracy from the fourth-order method, while the steps are taken using the accurate formula of fifth-order. However, it is also possible to implement other types of explicit Runge-Kutta methods (such as RK23, DOP853) or implicit methods (such as Radau, BDF) in this function. More details on these methods can be found at [23]. The default RK45 method seems to produce the most accurate results for the works of this thesis and thus it has been kept unchanged. There is another old function called \textit{odeint} [65] available in python for integrating a system of ordinary differential equations. However, the use of this very old \textit{odeint} function was avoided in these codes as it was creating unexpected overflow error [66] and hampering the calculation. The interactive graphical interface program for generating phase portraits, which was discussed in Chapter 3.3.4, was created using the Tkinter [47] module. \\

All the code files are uploaded to GitHub online repository under the link- \textbf{\url{https://github.com/tariquldipu/Masterarbeit}} \\

All the codes are available in two formats. Firstly, the codes are available in Anaconda Jupyter Notebook (as in .ipynb) format and are organized inside the Jupyter folder. The codes and their outputs are also saved in PDF formats inside this directory.
And secondly, the codes are also available as separate python files (as in .py format) and are organized in different folders as per their chapter names. \\ 

Some brief informations about each and every code file are also provided in the \textit{README.md} file inside the GitHub repository.
