Mathematical modeling is the process of describing a system using mathematical methods, language, concepts, and equations. Various mathematical models are being developed to understand, analyze, study, and predict different epidemiological and disease-inflicted situations in the human population. In this thesis paper, an attempt has been made to understand, study the scenario and predict the outcome of an epidemic caused by zombie infection on the human population. The development of this zombie epidemic model is based on the classical SIR model, which was originally proposed by Kermack-McKendrick in 1927. Based on this original SIR model, an analogous  SZR model, where the possibility of a zombie epidemic is considered, is studied here. The SZR model is analyzed through different mathematical processes to check and verify its stability. It is also being checked the possibility of human survival in the long run if such an epidemic occurs in real life through introducing and analyzing perturbation parameter in the model. A graphical interactive computer program is created using Python as part of this study, where it is possible to change and modify the parameter values of the model and simulate as well as visualize the change in outputs instantly. Moreover, in the subsequent step, a more complicated quarantine model, namely Modified SEZQR is considered, where the stability of the human population is studied through the introduction of perturbation parameter. A basic reproduction number is calculated based on this model and the condition for disease-free equilibrium is investigated through the impact of this perturbation parameter in bifurcating the model from instability to stability.