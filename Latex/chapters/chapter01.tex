\section{Background}

In recent times, mathematical modeling has been an important method in studying the spread of infectious diseases, contamination, and viral outbreaks. This process of mathematical modeling helps us to interpret the real-world scenario of the disease or outbreak in terms of mathematical language and equations. As such, it is possible to study the ongoing dynamics of the disease and predict the future flow or path of the disease, analyze and track its progression and finally take necessary steps to prevent or contain the outbreak in a more efficient and effective way. Through this modeling process, the whole disease inflicted scenario is translated in the form of mathematical equations based on the empirical observations, which are then later solved using different methods, parameters, and variables that enable researchers to understand the whole outbreak situation in a more comprehensive way and make informed decisions accordingly. \\


A lot of works have been done in modeling different types of outbreaks in the human population and the works on this field continue to flourish even further and wider. One of the most notable works in this field of studying epidemics through mathematical modeling can be traced back to the research paper \emph{A contribution to the mathematical theory of epidemics} [1], which was authored by Kermack and McKendrick in 1927. In this paper, the authors discussed about a deterministic modeling process, namely the SIR model, where an individual from a population would belong to one of the three categories- susceptible, infectious, or recovered. The model describes the transition of the disease through linear mass action law within these three compartments. This classical modeling approach of the SIR model has been taken as the basis of this thesis work. The modeling of the zombie infection follows an analogous process but also introduces and considers more variables and complications to make it more pragmatic and fitting for the real-world scenario. \\

 
In this thesis work, zombie infection is considered as the reason for causing the epidemic. In the pop culture (as in movies, thriller novels, video games) of recent times, the zombie apocalypse has become a popular phenomenon. As per the folklore definition of the Oxford dictionary [2], a zombie is believed to be something that is dead but turned alive partially through magic and is not fully aware of its surroundings. However, in today's pop cultures, zombies are being depicted as some kind of undead, cannibalistic, mindless creatures, that can attack and infect other healthy individuals and then also turn them into zombies and thus cause havoc within a population. So like any other kind of viral infection or infectious disease, a zombie has the ability to infect people and spread a disease. It is not the point of interest of this thesis work to investigate the origin of zombie infection, whatever that might be. Rather, it has been tried to study the dynamics through the process of mathematical modeling by considering the zombie epidemic as just another kind of infectious disease epidemic. In this paper, an attempt has been made to analyze the mathematical models of a likable zombie infection, introduce perturbation parameters to investigate the likelihood of a probable stability of such kind of epidemic through a scientific and mathematical approach. 

\pagebreak
\section{Related Work}

    A lot of research works have already been conducted in the field of mathematical modeling for the study of viral disease circulation among the population, both for humans and other types of animals. As already mentioned, the most notable pioneering work in this field comes from the SIR model developed by Kermack and McKendrick [1], where the authors had surmised a simple model of disease transmission within three compartments, namely from susceptible to infectious to recovered. Their work has turned into the ground basis for all kinds of future modeling approaches undertaken by different researchers in this field. An important work in understanding the process for developing mathematical equations and systems for epidemiological dynamics is described in detail in the book \emph{An introduction to mathematical epidemiology} [3]. Another comprehensive introduction can be found in the book \emph{Mathematical Epidemiology} [4], where the authors have discussed in detail regarding the preparation of mathematical models, along with the explanation for necessary mathematical concepts (such as calculus, differential equations, matrix algebra, probability) for understanding and simulating real-world outbreak scenario. Here, the authors have considered different real-world cases for infectious diseases such as Severe Acute Respiratory Syndrome (SARS), influenza, West Nile virus, etc. in explaining the process for developing a mathematical model. Also, to get some understanding of the concepts behind mathematical biology, the books \emph{Mathematical Biology I and II} [5], [6] are some of the important works in this field of study. \\


    However, in the simpler SIR model, a linear transformation of disease dynamics is considered and it is also assumed that an infected individual will receive immunity after a certain time. But this is not the actual case that happens in real-life scenarios. So, to understand the non-linearity, further studies are required. In this respect, \emph{Some epidemiological models with nonlinear incidence} [7] provide some important insights. In this paper, the authors have investigated different types of dynamics based on multiple equilibria points and related periodic solutions. Here, the authors have also introduced one more compartment, called Exposed in building up their model. In another work, S. Ruan and W. Wang [8] attempted to study the global dynamics of the epidemic through a nonlinear saturated mass action incidence rate. Also, for studying disease models without immunity, the works of H. W. Hethcote and P. van den Driessche [9] provided some important directions. More study on the SEIR (susceptible-exposed-infectious-recovered) model with nonlinear incidence rates can be found in the works of M. Y. Li and J. S. Muldowney [10]. \\


Numerous works have been done on modeling zombie epidemics. To get some primary idea, \emph{Teaching Mathematical Modeling Using the Zombie Apocalypse} [11] can be a good starting point. For Bayesian analysis of the zombie epidemic and some other diseases, the works of C. Witkowski and B. Blais [12] provides some directions.  Also, the book of A. Cartmel and R. J. Smith [13], provides a very rich and detailed overview for modeling zombies mathematically. Some other relevant papers for zombie epidemic modeling worth mentioning here are, the works of A. Alemi, M. Bierbaum [14], the works of E. Idu and R. Oladele [15], and the works of V. Tomotani [16]. \\


However, the two research works that have contributed the most in writing this thesis paper are \emph{Mathematical Modelling Of An Outbreak Of Zombie Infection} [17] and \emph{Perturbations in Epidemiological Models} [18]. The first paper [17] introduces the SZR model, and the quarantine model SEZQR (or SIZQR) and investigated whether it is possible for human to survive such a zombie epidemic. The second paper [18], took the analysis to next level by introducing perturbation parameters in the existing models and analyzed the survivability of the human population during such an epidemic. This paper inferred the likelihood of healthy human survival and termination of the zombie apocalypse with respect to the proper perturbation parameter. These works are the main basis and inspiration for analyzing the SZR model and making new modifications to the SEZQR model that will be presented in the upcoming chapters of this thesis paper.

\pagebreak
\section{Structure of the Thesis}

This thesis paper begins with a brief introduction to mathematical modeling for infectious diseases in chapter 1. Here, the process and importance of mathematical modeling are discussed, and also other related, relevant research works in this field are highlighted, that have served as the guiding directions for this thesis work. \\


In chapter 2, a brief introduction to the classical SIR model [1] has been given. This chapter explains the development of a simple model based on the three-class system, namely susceptible-infectious-recovered. This model describes the linear mass action law for the transmission of disease within a closed homogeneous population. \\


In chapter 3, a closer look has been given at the SZR model [18]. This model introduces zombies as the origin of the epidemic. Several mathematical analyses have been conducted on this basic model to check its feasibility and sustainability based on the perturbation parameter. An interactive graphical program is created using python to change the parameter values and simulate the changes in real-time in order to understand the dynamics in a more convenient way. The role of the perturbation parameter, as it changes values, on the model to test its stability has also been investigated here. \\


Finally, a more elaborate and complicated quarantine SEZQR model [17], [18] is studied in chapter 4. Some new modifications have been made on this model to give it a more pragmatic outlook and a new basic reproduction number is calculated based on this modified model. Later on, stability is checked for this modified model, and also the condition for the disease-free equilibrium is evaluated based on the perturbation parameter. Finally, the results of this modified model is compared with the original SEZQR model. Some special case scenarios have also been discussed at the end of this chapter.