In this thesis work, some attempts have been made to study the classical SIR model, and derive a more complicated yet pragmatic mathematical model from it to simulate and understand the real-world disease scenario. The zombie epidemic model that has been studied and analyzed here provides room for further studies on other types of infectious diseases that may cause epidemics in the human population. The simple zombie SZR model has been analyzed mathematically here and the graphical interface program that has been developed during this thesis work will provide scope for visualizing the model fast and easily in accordance with different sets of parameter values. This tool will ease the way to study and compare different model states under different circumstances. \\

Furthermore, in the more advanced and detailed modified SEZQR model, some more practical variables have been considered that may impact the zombie epidemic. Based on the modified model, a basic reproduction number is calculated and also the condition for initiating stability based on the perturbation parameter is derived. The value of the perturbation parameter depends on the human ability and capacity to contain and control the epidemic in the real world. During these analyses, some imaginary values of the conversion rate parameters were assumed. However, actual data need to be analyzed in the real world during the epidemic to get the most realistic estimates for these parameter values. Based on these parameter values, this modified model will be able to best predict and simulate the epidemic. \\

Even though many variables and situations have been considered in the modified model, there are still opportunities to include and consider many more situations that may occur in the real-world scenario. In the modified SEZQR model, the population nature is assumed to be homogenous. However, in further studies, there are scopes for analyzing the dynamics of the disease in a heterogeneous population. It can also be studied what may happen if the individuals are not contained within a closed region, rather they are moving within a wider and more diversified geographical area. The effects of modern medical treatment, vaccines, public awareness, weather, age, gender, food habit, quarantine facility, health condition, communication, etc. also play vital roles in understanding infectious disease dynamics. These are some of the example variables that may be considered in the future to expand and elaborate on the efficacy of this modified SEZQR model in studying new types of infectious diseases. However, in this thesis paper, the investigations have been kept confined within its theoretical dimensions. The example and comparisons are conducted to explain the model in a more comprehensive way. Finally, two best thought special cases are also considered that may alter the model in significant ways.
